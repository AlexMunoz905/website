\documentclass{article}
\usepackage{indentfirst, amssymb, amsmath, verbatim, latexsym, mathrsfs, algorithm, algorithmic, syntonly, float, graphicx, clrscode, hyperref}
\newcommand{\dotcup}{\ensuremath{\mathaccent\cdot\cup}}
\setlength{\parskip}{0pt}
\setlength{\parindent}{2em}
\linespread{1.4}
\title{Overlapping Criteria with Multinomial Model}
\author{Bowen Deng}
\date{}
\begin{document}
\maketitle
\begin{eqnarray}
I_i\thicksim \frac{p^{x_{k+1}+\cdots+x_n}}{A(n,k,l)}, i=1,2,\cdots,t.\\
x_i\in\{0,1\}. x_1+x_2+\cdots+x_n=l.\\
A(n,k,l)=\sum_{x_1+\cdots+x_n=l}p^{x_{k+1}+\cdots+x_n}.\\
p<k/2n.
\end{eqnarray}
\begin{displaymath}
\begin{split}
l\mathbb{E}C(l)&=\mathbb{E}|I_1\cap J_1|\\
&=\sum_{I_1}\Pr(I_1)\mathbb{E}(|I_1\cap J_1|\big| I_1)\\
&=\sum_{u=0}^{l\wedge k} p^{l-u}C_k^uC_{n-k}^{l-u}/A(n,k,l)\mathbb{E}(|I_1\cap J_1|) (I_1=\{1,2,\cdots,u,k+1,\cdots,k+l-u\})\\
&=\sum_{u=0}^{l\wedge k} p^{l-u}C_k^uC_{n-k}^{l-u}/A(n,k,l)\mathbb{E}\sum_{v=1,2,\cdots,u,k+1,\cdots,k+l-u}I(v\in J_1)\\
&=\sum_{u=0}^{l\wedge k} p^{l-u}C_k^uC_{n-k}^{l-u}/A(n,k,l)\sum_{v=1,2,\cdots,u,k+1,\cdots,k+l-u}\mathbb{E}I(v\in J_1)\\
&=\sum_{u=0}^{l\wedge k} p^{l-u}C_k^uC_{n-k}^{l-u}/A(n,k,l)\sum_{v=1,2,\cdots,u,k+1,\cdots,k+l-u}\Pr(v\in J_1)\\
&=\sum_{u=0}^{l\wedge k} p^{l-u}C_k^uC_{n-k}^{l-u}/A(n,k,l)(u\Pr(1\in J_1)+(l-u)\Pr(k+1\in J_1))\\
&=\sum_{u=0}^{l\wedge k} p^{l-u}C_k^uC_{n-k}^{l-u}/A(n,k,l)(u\frac{A(n-1,k-1,l-1)}{A(n,k,l)}+(l-u)\frac{pA(n-1,k,l-1)}{A(n,k,l)})\\
&=(\sum_{u=0}^{l\wedge k} up^{l-u}C_k^uC_{n-k}^{l-u})\frac{A(n-1,k-1,l-1)}{A^2(n,k,l)}+(\sum_{u=0}^{l\wedge k} (l-u)p^{l-u+1}C_k^uC_{n-k}^{l-u})\frac{A(n-1,k,l-1)}{A^2(n,k,l)}\\
&\triangleq s_1\frac{A(n-1,k-1,l-1)}{A^2(n,k,l)}+s_2\frac{A(n-1,k,l-1)}{A^2(n,k,l)}\\
s_1&=k\sum_{u=1}^l p^{l-u}C_{k-1}^{u-1}C_{n-k}^{l-u}\\
&=k(\sum_{u=1}^l C_{k-1}^{u-1}p^{l-u}C_{n-k}^{l-u})\\
&=kA(n-1,k-1,l-1)\\
s_2&=(n-k)p^2A(n-1,k,l-1)\\
l\mathbb{E}C(l)&=k(\frac{A(n-1,k-1,l-1)}{A(n,k,l)})^2+(n-k)(p\frac{A(n-1,k,l-1)}{A(n,k,l)})^2\\
\end{split}
\end{displaymath}
Note that
\begin{displaymath}
\begin{split}
\frac{\mathbb{E}(x_{k+1}+\cdots+x_n)}{\mathbb{E}(x_1+\cdots+x_k)}&=\frac{(n-k)\mathbb{E}x_{k+1}}{k\mathbb{E}x_1}\\
&=\frac{(n-k)\Pr(x_{k+1}=1)}{k\Pr(x_1=1)}\\
&=\frac{(n-k)\Pr(k+1\in I)}{k\Pr(1\in I)}\\
&=\frac{(n-k)pA(n-1,k,l-1)/A(n,k,l)}{kA(n-1,k-1,l-1)/A(n,k,l)}\\
&=\frac{(n-k)pA(n-1,k,l-1)}{kA(n-1,k-1,l-1)}\\
&\geqslant \frac{(n-k)p}{k}
\end{split}
\end{displaymath}
We expect $\rho\triangleq \frac{\mathbb{E}(x_{k+1}+\cdots+x_n)}{\mathbb{E}(x_1+\cdots+x_k)}$ to be smaller than $\frac{1}{2}$. Therefore, $p<\frac{k}{2n}$ is a reasonable assumption.\\
Lemma 1. If $f_1(x)$ is unimodal at $a$, $f_2(x)$ is unimodal at $b$, then $f_1(x)+f_2(x)$ takes maximum point in $[a,b]$.\\
Lemma 2. The following identities of $A(n,k,l)$ holds:
\begin{equation}
\begin{split}
A(n,k,l)&=A(n-1,k-1,l-1)+A(n-1,k-1,l)\\
&=pA(n-1,k,l-1)+A(n-1,k,l)
\end{split}
\end{equation}
\end{document}
