\documentclass{article}
\usepackage{indentfirst, amssymb, amsmath, verbatim, latexsym, mathrsfs, algorithm, algorithmic, syntonly, float, graphicx, clrscode, hyperref}
\newcommand{\dotcup}{\ensuremath{\mathaccent\cdot\cup}}
\setlength{\parskip}{0pt}
\setlength{\parindent}{2em}
\linespread{1.4}
\title{Report Note}
\author{Bowen Deng}
\date{}
\begin{document}
\maketitle
Why is cancer almost incurable?\\
We should first understand why cancer is hard to cure. Cancer is related to genome aberrations including gene mutations, copy number alterations and so on. Through these aberrations, the cancer cells can reproduce infinitely while other cells can't due to self-correction. This is the first reason why cancers are hard to kill: the huge amount and ability to reproduce. Another feature of cancer cells is they can spread to other tissues through blood circulation or lymphatic system.\\
We would like to understand the mechanism of these genome aberrations before making effective drugs to treat cancer.\\
Genome aberrations can be classified as:\\
\begin{itemize}
\item Passenger Mutation: neutral to cancer proliferation\\
\item Driver Mutation: Promote cancer cell to proliferate infinitely and diffuse
\end{itemize}
Finding out driver mutation is a key to understand cancer progression, and thus aid in designing effective drugs to treat cancer.\\
\\
How to study genome aberrations?\\
The advent of high-throughput sequencing technologies enables huge number of mutation profiles. Developing sequencing technologies is for biologists, and the job of statisticians is to mine useful information from these data. I'll introduce the previous studies on genome aberrations.\\
People first hope that cancers are caused by single genetic error and could be cured by fixing that specific problem. Much of that hope was based on the success of imatinib (Gleevec), a drug that was specifically designed to treat a blood cancer called chronic myeloid leukemia (CML). CML occurs because of a single genetic glitch that leads to the production of a defective protein that spurs uncontrolled cell growth. Gleevec binds to that protein, stopping its activity and producing dramatic results in many CML patients.\\
Unfortunately, the one-target, one-drug approach has not held up for most other types of cancer. Recent projects that deciphered the genomes of cancer cells have found an array of different genetic mutations that can lead to the same cancer in different patients. Then, based on the genetic profile of their particular tumor, patients could receive the drug or drug combination that is most likely to work for them.\\
The complexity of the findings appears daunting. Instead of attempting to discover ways to attack one well-defined genetic enemy, researchers now faced the prospect of fighting lots of little enemies. Fortunately, this complex view can be simplified by looking at which biological pathways are disrupted by the genetic mutations. Rather than designing dozens of drugs to target dozens of mutations, drug developers could focus their attentions on just two or three biological pathways. Patients could then receive the one or two drugs most likely to work for them based on the pathways affected in their particular tumors.\\
You might think of it like this: Imagine a thousand people from all across the United States travelling towards the front door of a single building in Chicago. How would you keep all of these people from entering the building?\\
If you had limitless resources, you could hire workers to go out and stop each person as he or she drove down the highway, arrived at the train station or waited at the airport. That would be the one-target, one-drug approach.\\
But if you wanted to save a lot of time and money, you could just block the door to the building. That is the pathway-based strategy that many researchers are now pursuing to design drugs for cancer and other common diseases.\\



\end{document}
