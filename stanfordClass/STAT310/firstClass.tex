\documentclass[a4paper,12pt]{article}
\begin{document}
Probability

TOLD TO US:\\
SET $\Omega$ (finite or countable)
\[
P(\omega) > 0, \sum_{\omega} P(\omega) = 1
\]
\[
	A \subseteq \Omega, P(A) = \sum_{\omega\in A} P(\omega)
\]

EX. Birthday Problem:
"How many people do we need to have chance $50\%$ that 2 or more have same birthday date?"
Say have n People and C categories.
What is $\Omega$?
\[
(\omega_1, \omega_2, \cdots, \omega_n), \omega_i \in \{1, 2, \cdots, C\}
\]
What is $P(\omega)$?
Try $P(\omega) = \frac{1}{C^n}$

What is A?
\[
A = \{\omega: \omega_i \neq \omega_j, \forall i, j\}
\]

\[
	P(A) = \sum_{\omega \in A} P(\omega) = \frac{1}{C^n} |A|
\]
where $|A| = C (C - 1) (C - 2) \ldots (C - n + 1)$

Answer 1:
\[
	P(A) = (1 - 1 / C)(1 - 2 / C) \ldots (1 - (n - 1) / C)
\]

Answer 2(HUMANS) Use $\log(1 - x) \tilde - X$
\[
	P(A) = \exp{\sum_{i = 1}^n log(1 - i / C)}
		\approx \exp{-\sum_{i = 1}^n i / C}
		= \exp{-\frac{(n 2)}{C}}
\]

Now set
\[
	e^{-(n 2)}{2} = \frac{1}{2} \rightarrow n = 1.2\sqrt{C} \approx 23
\]

Answer 3
$\log(1 - x) = -x + O(x^2)$
$-x - x^2 \leq \log(1 - x) \leq - x, 0 \leq x \leq \frac{1}{2}$

Theorem: if $n, C$ tend to $\infty$ so that
\[
\frac{n^3}{C^2} \rightarrow 0, \frac{(n 2)}{C} \rightarrow E
\]
\[
	P(A) \approx e^{-E}
\]

Problem $\$10$ How many people:
Do we need to have even odds to have triple match

More examples:
Put N points down at a rnorm in $[0, 1]^2$, put $\varepsilon$ Ball around each, what's chance cover?
We put probabilities on $\tau[0, 1]$
Manifolds

Half Way House:
\[
\Omega = (0, 1\}
\]
Work with intervals $(a_n, t_n\}$,
\[
	A = \cup_{i = 1}^n I_n, I_n
\]
Disjoint instances

Model for fair coin tossing:
\[
	\omega = \omega_1, \cdots
\]
\[
	\omega = \sum \frac{d_n(\omega)}{2^n}
\]
\[
	A = \{\omega: d_1(\omega)\}
\]
has $P(A) = \frac{1}{2}$, similarly,
\[
	P(A_1 = E_1, \cdots, A_n = E_n) = \frac{1}{2^n}
\]

Theorem (Bernoulli Weak Law of Large Numbers)
\[
	\forall \varepsilon > 0,
	P\{|\frac{d_1 + \cdots + d_n}{n} - \frac{1}{2}| > \varepsilon\} \rightarrow 0
\]

Proof: Define $\Omega_n(\omega) = 2 \times d_n(\omega) - 1$
Same to prove
\[
	\forall \varepsilon
	P\{|\frac{1}{n}\sum_{i = 1}^n \Omega_i| \rightarrow 0
\]
Note $\int_0^1 \Omega_i(\omega)\mathrm{d}\omega = 0$
\[
	\int_0^1 \Omega_i(\omega)\Omega_j(\omega) = \delta_{ij}
\]
So $\int_0^1 (\sum I_n)\mathrm{d}\omega = 0$, $\int_0^1 (\sum I_n)^2 \mathrm{d}\omega = n$
\[
	P\{|\frac{1}{n}\sum A_i| > 2 \varepsilon\}
\]
Apply Markov's inequality and we are done.

Lemma: if %$f:(0, 1) \rightarrow \mathbb{R}$,
%\[
%	f(\omega) \geqslant 0 
%\]
%Then $\forall a > 0$
%\[
%	P\{\omega f(\omega) \geqslant a\} \leqslant \frac{\int_0^1 f(\omega)\mathrm{d}\omega}{a}
%\]
%PF.
%\[
%	\int_0^1 f(\omega)\mathrm{d}\omega \geqslant \int_A f(\omega) \leqslant a P(A)
%\]


We want strong law: Borel's no.. number theorem:
\[
	\lim \frac{1}{n} \sum_{i = 1}^n \Omega_n(\omega)
\]
Problems: not tame
$w = .1111, w = 0110000111110$, limit doesn't exist

Def. $A \subseteq \Omega$ to be negligable
If $\forall \in A$ can be converted by countably many intervals of total length $<\varepsilon$.
eq. $x \in \Omega$ negligable, rationals in $(0, 1]$ negligable.


\end{document}
